\documentclass[12pt]{article}
\usepackage{amsmath}
\usepackage{graphicx}
\usepackage{hyperref}
\usepackage{listings}
\usepackage{color}

\title{Operating System Course Report - First Half of the Semester}
\author{B class}
\date{\today}

\begin{document}

\maketitle
\newpage

\tableofcontents
\newpage

\section{Introduction}
This report summarizes the topics covered during the first half of the Operating System course. It includes theoretical concepts, practical implementations, and assignments. The course focuses on the fundamentals of operating systems, including system architecture, process management, CPU scheduling, and deadlock handling.

\section{Course Overview}
\subsection{Objectives}
The main objectives of this course are:
\begin{itemize}
    \item To understand the basic components and architecture of a computer system.
    \item To learn process management, scheduling, and inter-process communication.
    \item To explore file systems, input/output management, and virtualization.
    \item To study the prevention and handling of deadlocks in operating systems.
\end{itemize}

\subsection{Course Structure}
The course is divided into two halves. This report focuses on the first half, which covers:
\begin{itemize}
    \item Basic Concepts and Components of Computer Systems
    \item System Performance and Metrics
    \item System Architecture of Computer Systems
    \item Process Description and Control
    \item Scheduling Algorithms
    \item Process Creation and Termination
    \item Introduction to Threads
    \item File Systems
    \item Input and Output Management
    \item Deadlock Introduction and Prevention
    \item User Interface Management
    \item Virtualization in Operating Systems
\end{itemize}

\section{Topics Covered}

\subsection{Basic Concepts and Components of Computer Systems}
    \subsubsection{Definisi dan Konsep}
    \hspace*{1cm}Sistem adalah kumpulan komponen yang saling berinteraksi dan bekerja sama untuk mencapai tujuan tertentu.
    
    Komputer adalah peralatan elektronik yang menerima masukan data, mengolah data, dan memberikan hasil keluaran dalam bentuk informasi. Komputer sendiri tidak berfungsi tanpa sistem komputer yang terintegrasi, karena sistem komputer yang lengkap meliputi semua komponen yang bekerja bersama-sama untuk menghasilkan informasi yang diinginkan12.
    
    Berdasarkan kedua definisi tersebut, Sistem komputer adalah kumpulan perangkat komputer yang saling terkait dan berinteraksi untuk menjalankan suatu aktivitas atau program menggunakan komputer. Sistem komputer terdiri dari tiga komponen utama: \textit {Hardware} (perangkat keras), \textit{Software} (perangkat lunak), dan \textit{Brainware} (manusia/pengguna)
    
Konsep Dasar Sistem Komputer

1. \textit {Hardware} (Perangkat Keras)
\textit {Hardware} adalah bagian fisik dari sistem komputer yang dapat disentuh. Komponen utama hardware meliputi:
\begin{itemize}
    \item \textit{Motherboard}: Papan induk yang menghubungkan semua komponen hardware.
    \item CPU (Central Processing Unit): Otak komputer yang melakukan pengolahan data.
    \item RAM (Random Access Memory): Memori sementara untuk data yang sedang diproses.
    \item Input/Output Unit: Perangkat untuk memasukkan dan mengeluarkan data.
\end{itemize}

2. \textit{Software} (Perangkat Lunak)
\textit{Software} adalah instruksi yang menjalankan hardware. Komponen utama software meliputi:

\begin{itemize}
    \item Sistem Operasi (OS): Program dasar yang menghubungkan pengguna dengan hardware dan mengelola sumber daya.
    \item Program Aplikasi: Program yang melakukan tugas spesifik seperti pengolah kata, peramban web, dan lain-lain.
\end{itemize}

\subsection{System Performance and Metrics}
This section introduces various system performance metrics used to measure the efficiency of a computer system, including throughput, response time, and utilization.

\subsection{System Architecture of Computer Systems}
Describes the architecture of modern computer systems, focusing on the interaction between hardware and the operating system.

\subsection{Process Description and Control}
Processes are a central concept in operating systems. This section covers:
\begin{itemize}
    \item Process states and state transitions
    \item Process control block (PCB)
    \item Context switching
\end{itemize}

\subsection{Scheduling Algorithms}
This section covers:
\begin{itemize}
    \item First-Come, First-Served (FCFS)
    \item Shortest Job Next (SJN)
    \item Round Robin (RR)
\end{itemize}
It explains how these algorithms are used to allocate CPU time to processes.

\subsection{Process Creation and Termination}
Details how processes are created and terminated by the operating system, including:
\begin{itemize}
    \item Process spawning
    \item Process termination conditions
\end{itemize}

\subsection{Introduction to Threads}
This section introduces the concept of threads and their relation to processes, covering:
\begin{itemize}
    \item Single-threaded vs. multi-threaded processes
    \item Benefits of multithreading
\end{itemize}

\subsection{File Systems}
File systems provide a way for the operating system to store, retrieve, and manage data. This section explains:
\begin{itemize}
    \item File system structure
    \item File access methods
    \item Directory management
\end{itemize}

\subsection{Input and Output Management}
Input and output management is key for handling the interaction between the system and external devices. This section includes:
\begin{itemize}
    \item Device drivers
    \item I/O scheduling
\end{itemize}

\subsection{Deadlock Introduction and Prevention}
Explores the concept of deadlocks and methods for preventing them:
\begin{itemize}
    \item Deadlock conditions
    \item Deadlock prevention techniques
\end{itemize}

\subsection{User Interface Management}
This section discusses the role of the operating system in managing the user interface. Topics covered include:
\begin{itemize}
    \item Graphical User Interface (GUI)
    \item Command-Line Interface (CLI)
    \item Interaction between the user and the operating system
\end{itemize}

\subsection{Virtualization in Operating Systems}
Virtualization allows multiple operating systems to run concurrently on a single physical machine. This section explores:
\begin{itemize}
    \item Concept of virtualization
    \item Hypervisors and their types
    \item Benefits of virtualization in modern computing
\end{itemize}

\section{Assignments and Practical Work}
\subsection{Assignment 1: Process Scheduling}
Students were tasked with implementing various process scheduling algorithms (e.g., FCFS, SJN, and RR) and comparing their performance under different conditions.

\subsection{Assignment 2: Deadlock Handling}
In this assignment, students were asked to simulate different deadlock scenarios and explore various prevention methods.

\subsection{Assignment 3: Multithreading and Amdahl's Law}
This assignment involved designing a multithreading scenario to solve a computationally intensive problem. Students then applied **Amdahl's Law** to calculate the theoretical speedup of the program as the number of threads increased.

\subsection{Assignment 4: Simple Command-Line Interface (CLI) for User Interface Management}
Students were tasked with creating a simple **CLI** for user interface management. The CLI should support basic commands such as file manipulation (creating, listing, and deleting files), process management, and system status reporting.

\subsection{Assignment 5: File System Access}
In this assignment, students implemented file system access routines, including:
\begin{itemize}
    \item File creation and deletion
    \item Reading from and writing to files
    \item Navigating directories and managing file permissions
\end{itemize}

\section{Conclusion}
The first half of the course introduced core operating system concepts, including process management, scheduling, multithreading, and file system access. These topics provided a foundation for more advanced topics to be covered in the second half of the course.

\begin{thebibliography}{99}

\bibitem{faaizah2023} Faaizah, N. (2023, 16 Oktober). Sistem Komputer: Pengertian, Bagian, Tujuan, dan Komponen Fisik. \textit{detikedu}, https://www.detik.com/edu/detikpedia/d-6983357/sistem-komputer-pengertian-bagian-tujuan-dan-komponen-fisik.

\bibitem{denning1989} Denning, P. J., Comer, D. E., Gries, D., Mulder, M. C., Tucker, A., Turner, A. J., \& Young, P. R. (1989). Computing as a discipline. \textit{Computer}, 22(2), 63-70.

\end{thebibliography}

\end{document}


\end{document}